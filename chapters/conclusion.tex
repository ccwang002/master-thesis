\chapter{Conclusions}
\label{c:conclusion}

Next-generation sequencing has been playing an important role in current
genome-wide study. However, an NGS analysis requires great computing resources
to complete and the result is not easily interpretable for biological
researchers and clinicians. In this study, an online NGS analysis platform,
BioCloud, was developed fully in house to help user automated the analysis
execution and provide summary report to understand the analysis result
interactively.  Additionally, user can manage their sequencing data sources and
reproduce previous analysis easily with the help of BioCloud.

BioCloud proposed a workflow for analysis execution by breaking the execution
into experiment design, analysis design, job queuing and summary report
generation. In experiment design, conditions are defined and group sequencing
sample that map the biological experiment conditions. In analysis design, user
specifies all parameters used in the analysis tool chain. Parameters are
grouped and optionally rendered based on the branch of tool execution user
chooses. Job queue handles the automatic execution of analyses, which user can
monitor the current execution status for ongoing jobs and get notified once the
execution completes. Summary report provides the overview of the analysis and
enables user to explore the results and compare outcomes of different
conditions in an interactive way. BioCloud introduces account system so
individual user can manage their own data and results naturally.

% account admin genome browser integration
Moreover, analysis output including genome alignment results can be further
integrated with the external genome browsers. User can compare their result
with numerous annotations available on Ensembl or UCSC database. The report and
results can be shared with others without BioCloud account easily, whose access
can be controlled by user via blocking all public access or generating new
access tokens. Some user with special privilege can access to the BioCloud's
admin interface and manage data on the whole platform.

% convenience and reproducibility
% Extensibility

With above functions BioCloud provides, BioCloud makes common NGS analyses more
convenient to manage and easier to comprehend. All user's NGS experiment
designs and commonly analyses can be maintained on BioCloud, an summary report
is provided for researcher to help them quickly understand the result and share
their results with others. For analysis pipelines that BioCloud currently does
not support, user can further develop new analysis pipeline on BioCloud by
extending the current available pipelines. In future, it can integrate with
other platforms to provide better functionality and more features. With the
adoption of BioCloud and reduce of effort in running routine analysis pipeline,
it can be anticipated that researchers will focus more on biological insight
from NGS analysis and development of novel analysis methods.

% vim: set textwidth=79 spell:
