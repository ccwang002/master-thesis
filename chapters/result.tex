\chapter{Results}
\label{c:result}

\section{Datasets}
\label{s:dataset}

%
% GSE52194 Breast cancer
% GSE52778 Airway Muscle and Asthma
%

Two RNA-Seq datasets and one DNA-Seq datasets from Gene Expression Omnibus
(GEO) were used for the demonstration of BioCloud. GSE52194
\cite{eswaran2012:transcriptomic} was a transcriptome profiling dataset of
human breast cancer. mRNA profiles of 17 breast tumor samples of three
different subtypes (TNBC, non-TNBC and HER2-positive) and normal human breast
organoids (epithelium) samples (NBS) were sequenced using Illumina HiSeq 2000
sequencer. Full sample list can be found in Table~\ref{tab:dataset-breast}.
Raw sequence reads were obtained from NCBI Sequence Read Archive (SRA), thus
their FASTQ file names were based on their SRA accession ID. In GSE52194 breast
cancer dataset, condition was defined to be the cancer subtype of the sample.

\begin{table}[!htbp]
    \caption[Experiment design of breast cancer GSE52194]{
        Experiment design of breast cancer dataset GSE52194.
    }
    \label{tab:dataset-breast}
    \centering
    \begin{threeparttable}
        \begin{tabular}{llr}
            \toprule
            Condition & Sample name & SRA ID and filename \\
            \midrule
            NBS      & NBS1      & SRR1027188 \\
            NBS      & NBS2      & SRR1027189 \\
            NBS      & NBS3      & SRR1027190 \\
            TNBC     & TNBC1     & SRR1027171 \\
            TNBC     & TNBC2     & SRR1027172 \\
            TNBC     & TNBC3     & SRR1027173 \\
            TNBC     & TNBC4     & SRR1027174 \\
            TNBC     & TNBC5     & SRR1027175 \\
            TNBC     & TNBC6     & SRR1027176 \\
            Non-TNBC & Non-TNBC1 & SRR1027177 \\
            Non-TNBC & Non-TNBC2 & SRR1027178 \\
            Non-TNBC & Non-TNBC3 & SRR1027179 \\
            Non-TNBC & Non-TNBC4 & SRR1027180 \\
            Non-TNBC & Non-TNBC5 & SRR1027181 \\
            Non-TNBC & Non-TNBC6 & SRR1027182 \\
            HER2     & HER2-1    & SRR1027183 \\
            HER2     & HER2-2    & SRR1027184 \\
            HER2     & HER2-3    & SRR1027185 \\
            HER2     & HER2-4    & SRR1027186 \\
            HER2     & HER2-5    & SRR1027187 \\
            \bottomrule
        \end{tabular}
    \end{threeparttable}
\end{table}


Another RNA-Seq dataset, GSE52778 \cite{himes2014:rnaseq}, is a transcriptome
profiling dataset of human airway smooth muscle (HASM). mRNA profiles of 4 male
white donors' HASM cells treated with four treatment conditions were sequenced
using Illumina HiSeq 2000 sequencer with Illumina TruSeq assay, four treatment
conditions being: no treatment (Untreated); treatment with Albuterol (Alb);
treatment with Dexamethasone (Dex); treatment with simultaneous Albuterol and
Dexamethasone (Alb\_Dex). Full sample list can be found in
Table~\ref{tab:dataset-airway}. Raw sequence reads were obtained from SRA thus
sample raw FASTQ files are renamed using their SRA accession ID.

\begin{table}[!htbp]
    \caption[Experiment design of GSE52778 airway muscle dataset]{
        Experiment design of GSE52778 airway muscle dataset.
    }
    \label{tab:dataset-airway}
    \centering
    \begin{threeparttable}
        \begin{tabular}{llr}
            \toprule
            Condition & Sample name & SRA Accession ID \\
            \midrule
            Untreated & N61311\_Untreaed   & SRR1039508 \\
            Untreated & N052611\_Untreated & SRR1039512 \\
            Untreated & N080611\_Untreated & SRR1039516 \\
            Untreated & N061011\_Untreated & SRR1039520 \\

            Dex       & N61311\_Dex        & SRR1039509 \\
            Dex       & N052611\_Dex       & SRR1039513 \\
            Dex       & N080611\_Dex       & SRR1039517 \\
            Dex       & N061011\_Dex       & SRR1039521 \\

            Alb       & N61311\_Alb        & SRR1039510 \\
            Alb       & N052611\_Alb       & SRR1039514 \\
            Alb       & N080611\_Alb       & SRR1039518 \\
            Alb       & N061011\_Alb       & SRR1039522 \\

            Alb\_Dex  & N61311\_Alb\_Dex   & SRR1039511 \\
            Alb\_Dex  & N052611\_Alb\_Dex\tnote{$\dagger$} & SRR1039515\tnote{$\dagger$} \\
            Alb\_Dex  & N080611\_Alb\_Dex  & SRR1039519 \\
            Alb\_Dex  & N061011\_Alb\_Dex  & SRR1039523 \\
            \bottomrule
        \end{tabular}
        \begin{tablenotes}
        \item[$\dagger$] This sample was excluded from the later-on analyses
            since its pair-end sequencing reads were mismatched.
        \end{tablenotes}
    \end{threeparttable}
\end{table}


% TODO: how deep is the sequencing depth of these WES samples?

The DNA-Seq dataset used in the demonstration was a human whole exome
sequencing done in our lab. Exome sequencing of 5 members from the same family
using Illumina HiSeq 2000 sequencer. The study aimed to find the common
variants shared in this family.



\section{Account registration and user dashboard}

A new user account was registered on BioCloud using email
\texttt{demo@biocloud.liang2.io}. Screenshots of the registration process are
shown in Figure~\ref{fig:biocloud-signup}, which resemble the process on common
websites. Most of the user inputs were sanity checked and validated, as shown
in Figure~\ref{fig:biocloud-signup}(b). After user completed the form, a
verification email was sent to the registered email address so as to make sure
all future email notifications can reach the user. A fixed message is displayed
on all pages of BioCloud if the user does not complete the email verification,
as shown in Figure~\ref{fig:biocloud-signup}(c).
Figure~\ref{fig:biocloud-signup}(d) shows the content of the verification
email, where an uniquely HMAC-based verification link was given. After clicking
on the link, user completed the verification and the fixed message for
verification was gone, as shown in Figure~\ref{fig:biocloud-signup}(e).

\begin{figure}[!p]
\centering
\includegraphics[width=1\textwidth]{images/biocloud_signup}
\caption[Account registration on BioCloud]{
    Account registration on BioCloud.
    (a) Registration form.
    (b) Password validation check.
    (c) Welcome screen after signup with action hint messages.
    (d) Verification email.
    (e) Welcome screen after email verification.
}
\label{fig:biocloud-signup}
\end{figure}


A user dashboard is available for user-related settings as shown in
Figure~\ref{fig:biocloud-dashboard}(b). User profile including user name and
authentication number can be updated here. Password can be changed via the link
in the password section in profile update, as shown in
Figure~\ref{fig:biocloud-dashboard}(b). For staff and superuser with special
permission, they can access to the admin interface of BioCloud in user
dashboard via an extra tab ``Admin'', as shown in
Figure~\ref{fig:biocloud-dashboard}(d). Admin interface will be introduced in
Section~\ref{s:biocloud-admin}.

\begin{figure}[!p]
\centering
\includegraphics[width=1\textwidth]{images/biocloud_dashboard}
\caption[Login and user dashboard on BioCloud]{
    Login and user dashboard on BioCloud.
    (a) Login form.
    (b) Password change form available through profile update.
    (c) Profile update form.
    (d) For staff and superuser's dashboard, an extra tab to the admin
    interface is shown.
}
\label{fig:biocloud-dashboard}
\end{figure}





\section{Data source discovery}

As a newly registered user, a message was always shown to hint user to attach
their data sources to BioCloud. User can find their specific data source folder
from menu Data Source \textrightarrow List, as shown in
Figure~\ref{fig:biocloud-data-source}(b). For the demo user, location of the
data source folder on BioCloud server was:

\begin{CVerbatim}[fontsize=\small]
/biocloud/data_sources/2/
\end{CVerbatim}

\vspace{-1em}\noindent
Upon new data sources being added, BioCloud updated its database by comparing
the data sources in record with that in current folder. All new discovered data
sources will be listed with guess of sample name and file type, as shown in
Figure~\ref{fig:biocloud-data-source}(a). User can pass the SHA2-256 checksum
value for each file computed locally. BioCloud computed the checksum of these
files and compared the value with that supplied by the user. User can check and
update the detail of their recorded data sources, as shown in
Figure~\ref{fig:biocloud-data-source}(b). BioCloud also guessed the strand for
pair-end FASTA/Q files. The information was recorded at the metadata column in
JSON format.

\begin{figure}[!tbp]
\centering
\includegraphics[width=1\textwidth]{images/biocloud_data_source}
\caption[Data source discovery on BioCloud]{
    Data source discovery on BioCloud.
    (a)
    (b)
}
\label{fig:biocloud-data-source}
\end{figure}





\section{Experiment design}

\begin{figure}[!tbp]
\centering
\includegraphics[width=1\textwidth]{images/biocloud_experiment_design}
\caption[Experiment design on BioCloud]{
    Experiment design on BioCloud.
    (a) Description Markdown editor in full screen.
    (b) Upper half of the experiment design form.
    (c) Condition assignment of samples.
    (d) Condition design.
}
\label{fig:biocloud-experiment-design}
\end{figure}

\begin{figure}[!tbp]
\centering
\includegraphics[width=1\textwidth]{images/biocloud_experiment_sample_condition}
\caption[Experiment design on BioCloud (cont'd)]{
    Experiment design on BioCloud (cont'd).
    (a)
    (b)
    (c)
    (d)
    (e)
}
\label{fig:biocloud-experiment-sample-condition}
\end{figure}



% general description
Experiment design groups selected samples into user defined conditions to
represent the biological experiment design. BioCloud will warn the user if one
try to create experiment without adding any data source. All experiment related
functions are under the menu item ``Experiment''. Here the demo user added all
datasets mentioned in Section~\ref{s:dataset}. When the user created a new
experiment, one was guided to fill in experiment description, select data
sources, create conditions, assign conditions step by step, as shown in
Figure~\ref{fig:biocloud-experiment-design}. User first filled in the
experiment name and description. Experiment description was written using the
Markdown syntax and an editor based on SimpleMDE was provided, as shown in
Figure~\ref{fig:biocloud-experiment-design}(a). Headings, enumerated lists, and
hyperlinks can be created to help user collect all the related information. The
rendered description can be live previewed in full screen side-by-side mode.

% sample selection
To actually construct the experiment design, user first started with sample
selection, as shown in Figure~\ref{fig:biocloud-experiment-design}(b). By
default the simple mode of selection sample was shown, where data sources of
same sample names were selected together and the default sample name was used.
Selected samples were simultaneously shown in the condition design section
below (Figure~\ref{fig:biocloud-experiment-design}(c)) with the help of
reactive programming. All the elements in the experiment design form shares the
same underlying data structure. Therefore, as a sample was selected, all its
data sources are marked as selected and Vue.js updated all the DOMs that
depended on the selected data sources.

% simple and advanced mode
An advanced mode of selection mode was provided so user can have detail control
on the inclusion of samples, as shown in
Figure~\ref{fig:biocloud-experiment-sample-condition}(b). When it is enabled
data sources were listed one by one and their sample names become modifiable.
User can select one strand of a pair-end sample, or override the existed sample
name of data sources. Data sources can be filtered based on file name, file
type, and sample name.

% condition
By default samples are of condition ``(All)'', a special condition implying the
sample does not belong to any given condition. Currently no pipeline requires
the special condition but it is implemented so custom pipeline can let user
specify experiment-wise input. By default there were two conditions: Control
and Testing. User can remove, edit, and add conditions, as shown in
Figure~\ref{fig:biocloud-experiment-design}(d). They were dynamically rendered
as available options of selected sample's conditions, as shown in
Figure~\ref{fig:biocloud-experiment-design}(e). Note that it is considered safe
and there will be no data loss if user later modifies the condition name.
Order of conditions was kept in the database. In analysis, condition specified
later was compared with those specified earlier for each condition pair.
Generally, one should put control or normal condition as the first condition.
In the figure, breast cancer dataset defined conditions: NBS, TNBC, Non-TNBC,
HER2 in order, where NBS was the normal condition of the dataset. As user was
designing the conditions, a preview of the current experiment was displayed in
the last section, as shown in
Figure~\ref{fig:biocloud-experiment-sample-condition}(c).

%experiment list
User can view one's submitted experiments at the experiment list view, as shown
in Figure~\ref{fig:biocloud-experiment-sample-condition}(e). Detail of the
experiment can be found by clicking the link of each experiment in the
experiment list view, as shown in
Figure~\ref{fig:biocloud-experiment-sample-condition}(d). Users are not allowed
to modify the experiment conditions, otherwise existed analyses would have
uncertain experiment design. However, one can build a new experiment based on
an existed experiment design.



\section{Analysis design}

\begin{figure}[!p]
\centering
\includegraphics[width=1\textwidth]{images/biocloud_analysis_design}
\caption[Analysis design on BioCloud]{
    Analysis design on BioCloud.
}
\label{fig:biocloud-analysis-design}
\end{figure}



Analysis binds an experiment design and executes one of the available pipelines
with given genome reference and parameters. First user chose the pipeline to
execute, as shown in Figure~\ref{fig:biocloud-analysis-design}(a). It was a
portal which gathered all currently available pipelines defined in the BioCloud
source code. In Figure~\ref{fig:biocloud-analysis-design}, the demo user
selected the Cufflink-based RNA-Seq pipeline. User was then guided to a new
dedicated page for the selected analysis design form. Similar to experiment
design, user can specify name and description in Markdown syntax for the
new analysis.

Some parts of the analysis design form were the same for all analysis
pipelines. All analysis pipelines require genome reference and experiment
design as input. Available genome references were defined by staff or superuser
via the BioCloud admin interface, whose name and provider combined as the
option name as shown in Figure~\ref{fig:biocloud-analysis-design}(b). Here the
human genome reference hg38 provided by UCSC was chosen. User continued to
select the experiment design in use. User can select by the name of one's own
experiment designs, whose conditions and samples will be dynamically rendered
once user made a choice, as shown in
Figure~\ref{fig:biocloud-analysis-design}(c). A link to the experiment detail
view was also provided for more information. Here the human airway smooth
muscle dataset was used.

Finally, rest of the analysis design form was specially defined for
Cufflink-based RNA-Seq pipeline. As described in
Section~\ref{s:rnaseq-pipeline}, there are three available genome aligners in
this pipeline. Each of the aligner has different option sets. Therefore, as
user selected one aligner, only the corresponding options were shown. For
example, in Figure~\ref{fig:biocloud-analysis-design}(d), user changed the
aligner from STAR to Tophat2 and different options were displayed.



\section{Job queue monitoring}

\begin{figure}[!p]
\centering
\includegraphics[width=1\textwidth]{images/biocloud_job_monitor}
\caption[Job monitoring on BioCloud]{
    Job monitoring on BioCloud.
}
\label{fig:biocloud-job-monitor}
\end{figure}



% analysis list
% execution detail
% report link and options
% email notification

User could find all executed, queuing and currently running analyses at one's
analysis list view, as shown in Figure~\ref{fig:biocloud-job-monitor}(a). By
clicking on the name of the analysis, user was guided to its detail page. If
the analysis is currently running, its detail will be live updated to reflect
the current status of its execution stages. Some of the stages may be marked as
skipped if user opted out their execution during analysis design. For example,
Figure~\ref{fig:biocloud-job-monitor}(b) shows an analysis currently running
the Cufflinks stage. QC stage was skipped thus this analysis did not perform
the quality check. Stages behind the currently running stage, e.g. Cuffdiff
stage in the figure, were marked as waiting.

As the analysis execution ended, user was notified by mail with the link to the
analysis detail page, as shown in Figure~\ref{fig:biocloud-job-monitor}(d). If
the analysis execution was successful, the summary report was generated and its
link would be shown in its analysis detail page.
Figure~\ref{fig:biocloud-job-monitor}(c) showed the successful execution of the
analysis described above. Summary report was generated and its canonical link
was given below the execution detail. When user clicked on the link, one would
be redirect to the report link with HMAC-based access token. During logged in,
both links will maintain valid until user modifies one's authentication number
at the user profile page, which invalidate the current HMAC-based access token
while the canonical link will always be accessible. By default, all reports and
results were restricted to private access only. User can opt the current
analysis to be public by changing the ``is public'' setting shown in
Figure~\ref{fig:biocloud-job-monitor}(c).



\section{Summary report}

Summary report was available for successfully executed analysis. It can be
viewed on the BioCloud by accessing the report link shown in
Figure~\ref{fig:biocloud-job-monitor}(c). The summary report using human airway
smooth muscle (HASM) dataset and STAR and Cufflinks RNA-Seq analysis pipeline
was shown as the example in the following sections.

\begin{figure}[!tb]
\centering
\includegraphics[width=1\textwidth]{images/report_home}
\caption[Result summary report]{
    Result summary report.
}
\label{fig:report}
\end{figure}




Home page of the summary report, shown in Figure~\ref{fig:report}, provides the
statistics and general information about the analysis, including time usage,
owner, and parameters for the pipeline. Here 15 samples (30 data sources) were
used and 4 conditions, Untreated, Dex, Alb, and Alb\_Dex were defined.
Parameters of the analysis was shown in the same report section. For this
analysis, UCSC hg38 genome reference was used. Sidebar of the summary report
collected outputs from different stages of the pipeline. Main stages included
quality check (FastQC), genome alignment (STAR), expression level inference
(Cufflinks), and differential expression analysis (Cuffdiff).


\subsection{Quality control}

\begin{figure}[!tb]
\centering
\includegraphics[width=1\textwidth]{images/report_qc}
\caption[Quality check page of report]{
    Quality check page of report.
}
\label{fig:report-qc}
\end{figure}


Following the analysis order, first we checked the sequencing quality of the
HASM dataset. Main output of FastQC during the quality check stage is shown in
Figure~\ref{fig:report-qc}. Basic statistics showed the sequencing throughput
for each sample was around 20M reads and 63 bp long. Some of the pair-end
samples showed imbalanced number of reads for its forward and reversed reads,
including N052611\_SRR1039513. This may indicate data corruption during file
transferring via network. However, the data sources have been doubled
downloaded and their checksum were verified. Since the sequencing data were
obtained from the public SRA repository, some raw sequence reads with subpar
quality may have been filtered before submission. Figure~\ref{fig:report-qc}(b)
shows the plot of per base quality.  Only the median quality of all reads at
the given position was displayed. Reads of many reversed end samples
experienced quality drop around 10th and 39th bp, while most of the reads
remained in the good quality zone. One can investigate the sequencing
instrument if this quality drop pattern become more severe. The plot is
interactive so user can mark only data sources of interest and zoom into
particular region.  User can also export the current plot as a static file in
PNG, JPEG, or SVG format.


\subsection{Genome Alignment -- STAR}

\begin{figure}[!tb]
\centering
\includegraphics[width=1\textwidth]{images/report_align}
\caption[STAR alignment plot in summary report]{
    STAR alignment plot in the summary report of HASM dataset.
    (a) Alignment expressed in the number of reads.
    (b) Alignment expressed in the percentage of total input reads.
}
\label{fig:report-align}
\end{figure}


The pair-end sequencing samples of HASM dataset were first aligned to human
genome using STAR aligner. The alignment results are shown in
Figure~\ref{fig:report-align}. One can view the alignment result in either
number of reads or percentage. Reads of all samples were at least 90\% uniquely
aligned. Samples of the same condition were grouped together with same
underlying background color. Around 3\% of reads were mapped to multiple
locations and around 2\% of reads failed to align to genome due to being too
short. By toggling off the display of uniquely mapped reads, one can be able to
view a clearer breakdown of read proportion of rest of the alignment types.
All samples were well aligned to the human genome. One can find the exact read
counts in the table shown in Figure~\ref{fig:report-align-table}.

\begin{figure}[!tb]
\centering
\includegraphics[width=1\textwidth]{images/report_align_table_num_read}
\caption[STAR alignment table in summary report]{
    STAR alignment table summary in the report.
}
\label{fig:report-align-table}
\end{figure}



During alignment, many files including alignment result, its index, and log
files were generated,  whose access link were collected in the last section of
the report page. As shown in Figure~\ref{fig:report-align-link}. The link not
only enables user to download the output files but also helps user share their
result with others. If the report is flagged as public, these links can be
accessed by other users and external tools such as genome browser.

\begin{figure}[!tb]
\centering
\includegraphics[width=1\textwidth]{images/report_align_links}
\caption[STAR alignment original output file links in summary report]{
    STAR alignment original output file links in summary report.
}
\label{fig:report-align-link}
\end{figure}



\subsection{Cuffdiff}

\begin{figure}[!p]
\centering
\includegraphics[width=1\textwidth]{images/report_cuffdiff}
\caption[Cuffdiff stage of summary report]{
    Cuffdiff stage of summary report.
}
\label{fig:report-cuffdiff}
\end{figure}



After aligned to the genome reference, the gene expression of all samples were
statistically inferred by Cufflinks and compared across conditions by Cuffdiff.
Except for access links to the original outputs of these two tools. BioCloud
generated an overview of expression across conditions, as shown in
Figure~\ref{fig:report-cuffdiff}(a) and (b).
Figure~\ref{fig:report-cuffdiff}(a) showed the gene expression distribution of
samples per condition and the pairwise distribution matrix. Most distributions
did not skew to one of the condition, implying the experiment throughput of
different samples have been correctly by the Cufflinks statistical framework.
Figure~\ref{fig:report-cuffdiff}(b) showed the volcano plot of asthma
medication treated samples with untreated samples respectively. None of the
gene expression after being treated has been significantly changed, where the
$p$-value has been multiple test adjusted.

User can further investigate the differential expression data by downloading
the original output of Cuffdiff. Output at this stage takes much less disk
space and computing resources in comparison to the early stages in NGS
analysis. We extracted a subset of gene expression that were mentioned in the
original study \cite{himes2014:rnaseq} and plot their expression level in
different conditions using cummeRbund, as shown in
Figure~\ref{fig:report-cuffdiff}(c). The trend of the gene expression were
coherent to that of the original study.



\subsection{Integration with external genome browsers}

\begin{figure}[!p]
\centering
\includegraphics[width=1\textwidth]{images/result_genome_browser}
\caption[Result integration with genome browser]{
    Result integration with genome browser.
}
\label{fig:result-genome-browser}
\end{figure}



Raw output by each tools were provided via separate download links in the last
section of each report page. These raw output links can be further integrated
with popular genome browsers like Ensembl and UCSC genome browser. Note that
Current Ensembl v84 release does not support HTTPS connections to remote BAM
files due to an older version of samtools was used. However, in the next
release the HTTPS connection will be supported and all operations on UCSC
genome browser can be applied to Ensembl genome browser in a similar fashion.
Here we take the alignment BAM files from breast cancer dataset and display the
alignment on UCSC genome browser as example, shown in
Figure~\ref{fig:result-genome-browser}. User specified custom data track by the
following format:

\begin{CVerbatim}[fontsize=\small]
track type=bam name="NBS1"
bigDataUrl=https://biocloud.cgm.ntu.edu.tw/access/<auth key>/result/
STAR/NBS1/Aligned.sortedByCoord.out.bam
\end{CVerbatim}

\vspace{-1em}\noindent
Here the data track was specified for NBS1 sample from the breast cancer
dataset. User can then view their alignment on UCSC genome browser in the same
way as on desktop genome browser tools like IGV. Genome browser didn't download
the whole BAM file before displaying the alignment. Instead, it utilized the
BAM index to look up the alignment data of the current viewing genomic region.
Many file formats other than BAM are supported, including GTF and VCF.



\section{Admin interface}
\label{s:biocloud-admin}

\begin{figure}[!p]
\centering
\includegraphics[width=1\textwidth]{images/biocloud_admin}
\caption[Admin interface of BioCloud]{
    Admin interface of BioCloud.
}
\label{fig:biocloud-admin}
\end{figure}




As shown in Figure~\ref{fig:biocloud-dashboard}(d), staff and superuser will
have the special permission to access the admin interface of BioCloud. The
homepage of the BioCloud admin is shown in Figure~\ref{fig:biocloud-admin}(a).
In admin interface, user is allowed to add genome references, diagnose tasks in
the analysis job queue, and modify all user's data.
Figure~\ref{fig:biocloud-admin}(b) shows the section managing experiments where
one can find BioCloud defines many hidden fields including owner and created
date of a experiment. These fields can be further change by clicking into the
desired experiment, as shown in Figure~\ref{fig:biocloud-admin}(c). Noted that
some of input validations were disabled in admin interface so user should be
responsible and careful for all the actions performed here.

% vim: set textwidth=79 spell:
