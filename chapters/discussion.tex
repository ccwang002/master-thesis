\chapter{Discussions}
\label{c:discussion}


\section{Data source upload}

BioCloud does not offer the ability to upload data sources from local to
BioCloud's data storage since the uploading is extremely hard to implement
correctly and few people use uploading function anyway. Uploading a pair-end
sample of 22M 63 base pair long reads takes about 3.5 hours for normal ADSL
connection or 20 minutes for education network to complete\footnote{The upload
speed is given 1MB/s for normal ADSL connection and 10MB/s for education
environment; The pair-end sample of 22M reads with 63 base pair consumes total
12 GiB disk space}. Upload a dataset having 10 samples may take days for users
of low connection speed. Though a lot of HTTP-based uploading frameworks exist,
they are only suitable for uploading few hundreds of MBs at a time. Moreover,
most users may not upload any data sources in their lifetime since their raw
sequencing data exist on the server in the very first place. BioCloud supports
reading soft-linked data sources so putting data sources under BioCloud data
storage structure does not necessarily increase the total disk usage. When data
source transferring is needed, one can use tools such as \texttt{rsync},
\texttt{sftp} and \texttt{ftp} to help transfer large files rather than using
HTTP-based solutions. After data sources appear in the user's data source
directory, they can be discovered and managed by BioCloud.



\section{Pipeline extension}
\label{s:pipeline-extension}

% \begin{lstlisting}
% class Hello:
%
%     def __init__(self, a, b):
%         self.x = a
%         self.y = b
%
%     @classmethod
%     def from_str(cls, raw_str):
%         return cls(*raw_str.split('\t', 1))
% \end{lstlisting}


\section{Collaboration with other frameworks}

% vim: set textwidth=79:
