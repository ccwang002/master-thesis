\begin{acknowledgementszh}

感謝指導老師莊曜宇教授,提供全方面的資源支持我研究,給予我充份的空間與機會去嘗
試各種新的技術與想法。謝謝蔡孟勳老師提供我諸多研究點子,及參與實驗室合作案們的
機會。謝謝賴亮全老師在生物、統計上的建議。感謝盧子彬老師提供許多研究發表機會,
申請期間寶貴的建議,改變了我研究生涯。感謝在北京微軟的張益肇博士與許燕老師,你
們開拓了我的研究視野。謝謝張博士教導了我大尺度的領域發展觀察、許老師在北京生活
與學習無時的照顧。我經常懷念夥伴們寫程式時的歌唱、一周一百小時的研究時光。

謝謝學長姐們平日在實驗室的照顧。謝謝建樂學長從大學專題生引領我 miRNA 定序分析
起,到指導我的網站開發與碩論回饋;謝謝嘉珊學姐擔任生物問題的求救對象;謝謝鈺喬
學長雙方面的指導以及臉書上的加油打氣。謝謝承桓學長對伺服器們的維護,讓分析都能
順利完成。也謝謝翔瀚、恆元、勁廷、偉安、沂芳、家郁、耀瑋不論是分析、研究、申請
還是日常的交流與抒壓。 也感謝社群朋友們技術分享與生涯規劃的諸多啟發,以及讓我犯
錯冒險的機會。感謝 R 社群的 Wush、c3h3、chiawei、yen 引領我進入開源的大家庭。謝
謝 Python 社群的 yyc、TP、Tim、Andy、Michelle、Keith。沒有你們,我沒有今日的技
術、能力、想法、以及工程師的世界。也謝謝家人、攝影社朋友、高中同學們一路上的支
持。

加入實驗室團隊轉瞬已四年時光,再次感謝所有指導老師、所有實驗室同仁以及朋友的幫
忙。

\hfill 王亮博\quad2016 年 7 月
\end{acknowledgementszh}

%\begin{acknowledgementsen}
%\end{acknowledgementsen}
% vim: set tw=79:
